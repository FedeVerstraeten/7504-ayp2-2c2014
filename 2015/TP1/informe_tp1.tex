% ------------------------------------------------------------------------- //
% Facultad de Ingeniería de la Universidad de Buenos Aires
% Algoritmos y Programación II
% 1er Cuatrimestre de 2015
% Trabajo Práctico 1: Recursividad
% Cálculo de DFT
% 
% informe_tp1.tex
% Informe
%
% Para compilar:
% $: pdflatex informe_tp1
% ------------------------------------------------------------------------- //

% ---- Preamble ----
\documentclass{article}


% ---- Packages ----
\usepackage{amsmath} % Advanced math typesetting
\usepackage[utf8]{inputenc} % Unicode support (Umlauts etc.)
\usepackage[spanish]{babel} % Change hyphenation rules
\usepackage{hyperref} % Add a link to your document
\usepackage{graphicx} % Add pictures to your document
\usepackage{float} % For 'H' figure position option. Stricter than 'h!'
%\usepackage{listings} % Source code formatting and highlighting
\usepackage{listingsutf8} % Source code formatting and highlighting in UTF-8
\usepackage[a4paper]{geometry} % Page size options
  \geometry{tmargin=3cm,bmargin=3cm,lmargin=2cm,rmargin=2cm} % Margins
\usepackage{fancyhdr} % Add head­ers and foot­ers
  \setlength{\headheight}{14pt} % Needs to be 13.6pt or more
  \pagestyle{fancy}
  %\fancyhf{} % Clear the default settings
  %\lhead{left header content}
  %\chead{middle header content}
  %\rhead{right header content}
  %\lfoot{left footer content}
  %\cfoot{middle footer content}
  %\rfoot{right footer content}
\usepackage{color} % Colours
  %red, green, blue, yellow, cyan, magenta, black, white
  \definecolor{mygreen}{RGB}{28,172,0} % color values Red, Green, Blue
  \definecolor{mylilas}{RGB}{170,55,241}
  \definecolor{mygray}{rgb}{0.5,0.5,0.5}
  \definecolor{mymauve}{rgb}{0.58,0,0.82}
  \definecolor{myblue}{rgb}{0.33,0.33,0.99}
\hypersetup{ % Remove hyperlink borders
  pdfborder={0 0 0}
}
% ------------------

% ---- Formato para código fuente ----
\lstset{
    language=C++,
    basicstyle=\color{red},
    breaklines=true,%
    morekeywords={matlab2tikz},
    keywordstyle=\color{blue},%
    morekeywords=[2]{1}, keywordstyle=[2]{\color{green}},
    identifierstyle=\color{black},%
    stringstyle=\color{mygreen},
    commentstyle=\color{mygray},%
    showstringspaces=false,%without this there will be a symbol in the places where there is a space
    numbers=left,%
    numberstyle={\tiny \color{mygray}},% size of the numbers
    numbersep=9pt, % this defines how far the numbers are from the text
    emph=[1]{for,end,break},emphstyle=[1]\color{blue}, %some words to emphasise
    emph=[2]{word1,word2}, emphstyle=[2]{style},    
    inputencoding=utf8/latin1, % Para código con tildes y otros caracteres
}
% ------------------------------------


% --------------------------------------------------------------------------- %
% ---- Comienzo del documento ----
\begin{document}

% ---- Carátula ----
\title{Algoritmos y Programación II\\
       TP1: Recursividad}
\author{Bourbon, Rodrigo\\
        Carreño Romano, Carlos Germán\\
        Sampayo, Sebastián Lucas}
\date{Primer Cuatrimestre de 2015}
\maketitle

\begin{center}
  \includegraphics[width=0.5\paperwidth]{Imagenes/logo_fiuba_HD}
  %\rule[depth]{width}{height}
  \rule[0.5ex]{0.8\paperwidth}{0.1pt}
\par
\end{center}

\pagenumbering{gobble} % Don't number this page
% ------------------

% ---- Encabezado ----
\lhead{Algoritmos y Programación II - TP1 - FIUBA}
% --------------------

% ---- Tabla de contenidos ----
\newpage{}
\vfill{}
\tableofcontents{}
\vfill{}
\newpage{}
% -----------------------------
% ------------------------------------
\pagenumbering{arabic} % Do number this page, arabic numbers


\section{Objetivos}
  Ejercitar técnicas de diseño, análisis, e implementación de algoritmos recursivos.

\section{Introducción}
  Explicar un poco que es la FT, la DFT y la FFT.

\section{Standard de estilo}
  Adoptamos la convención de estilo de código de Google para C++, salvando las siguientes excepciones:
  \begin{itemize}
    \item Streams: utilizamos flujos de entrada y salida
    \item Sobrecarga de operadores
  \end{itemize}
  \url{https://google-styleguide.googlecode.com/svn/trunk/cppguide.html#Naming}

\section{Diseño del programa}
  Explicar a grandes rasgos como funciona el programa, diagrama en bloques.
  -> Leer de la entrada a vector, rellenar con ceros, transformar, imprimir vector.

\section{Opciones del programa}
  El programa se ejecuta en línea de comandos, y las opciones que admite (sin importar el orden de aparición) son las siguientes:
  \begin{itemize}
    \item[] \textit{nombre largo} (\textit{nombre corto}): \textit{descripción}
    \item \texttt{--input} (\texttt{-i}): 
    \item \texttt{--output} (\texttt{-o}):
    \item \texttt{--method} (\texttt{-m}):
  \end{itemize}

\section{Métodos de la Transformada}
  como fue implementado dft y fft, funciones genéricas, máscaras, complejidad temporal, espacial, etc.
  \subsection{FFT}
    \subsubsection{Complejidad Temporal}
      Para estudiar el costo temporal de esta implementación ---$T(N)$--- se analizó
    cada línea de código de la función \textit{calculate\_fft\_generic()}.\par
    Al principio, todas las sentencias son de orden constante hasta que
    aparece el primer ciclo:
      %copiar una vez tengamos la versión final
    \begin{lstlisting}
    \end{lstlisting}
      Las únicas expresiones que ofrecen cierta duda de que su coste sea 
    constante son las últimas ---constructores de N/2 elementos. Sin embargo,
    al ver la implementación de dicho constructor no quedan dudas, ya que solo
    consiste en una comparación, una asignación, y una llamada a \textit{new}:
      % copiar código función Vector(int)
    \begin{lstlisting}
    \end{lstlisting}
      Luego se tiene un ciclo de N/2 iteraciones cuyas operaciones en cada caso
    son de orden constante, con lo cual el orden de este ciclo es $\mathcal{O}(N/2)$.
      %copiar una vez tengamos la versión final
    \begin{lstlisting}
    \end{lstlisting}
      A continuación encontramos las llamadas recursivas. Dado que el tamaño
     de la entrada se reduce a la mitad, tenemos 2 llamadas de coste $T(N/2)$.
      %copiar una vez tengamos la versión final
    \begin{lstlisting}
    \end{lstlisting}
      Finalmente, se tiene un ciclo de N iteraciones cuyas operaciones en cada 
    caso son de orden constante, produciendo un coste de $\mathcal{O}(N)$.
      %copiar una vez tengamos la versión final
    \begin{lstlisting}
    \end{lstlisting}
      De esta forma, agrupando estos resultados parciales, se puede escribir la
     ecuación de recurrencia para este algoritmo:
    \begin{align*}
      T(N) &= \mathcal{O}(1) + \mathcal{O}(N/2) + 2T(N/2) + \mathcal{O}(N) \\
      T(N) &= 1 + N + 2T(N/2) \\
    \end{align*}
    \begin{equation*}
      \boxed{T(N) = 2T(N/2) + N}
    \end{equation*}
      Como se puede ver, es posible aplicar el teorema maestro, definiendo:
    \begin{align*}
      a &= 2 \geq 1 \\
      b &= 2 > 1\\
   f(N) &= N
    \end{align*}
      Utilizando el segundo caso del teorema:
    $$ \exists\,k \geq 0 \quad / \quad N \;\epsilon\; \Theta (N^{\log_b (a)} \log^k (N)) $$
    $$ \Rightarrow T(N)\;\epsilon\;\Theta (N^{\log_b (a)} \log^{k+1} (N)) $$
      Es fácil ver que con $k=0$ dicha condición se cumple, por lo tanto
    el resultado final es:
    $$ \boxed{T(N)\;\epsilon\;\Theta (N \log N)} $$
      Este resultado es coherente, ya que el algoritmo utiliza la técnica de 
    "divide y vencerás" y la recurrencia es análoga al caso del conocido 
    \textit{MergeSort}.




\section{Estructura de archivos}

\section{Compilación}
Como se compila

\section{Casos de prueba}
los q aparecen en la especificación del tp, mostrar capturas de pantalla
de la consola ejecutando todo

\section{Código}

% Plantilla de figura:
%  \begin{figure}[H]
%  \begin{centering}
%  \includegraphics[width=0.75\textwidth]{Imagenes/SDR.jpg}
%  \par\end{centering}
%  \caption{Sintonizador de radio digital.}
%  \end{figure}

% Plantilla de código fuente:
%    \lstinputlisting{code.cc}

\section{Enunciado}

\end{document}
